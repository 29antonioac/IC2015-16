\documentclass[a4paper, 11pt, titlepage]{article}
%Comandos para configurar el idioma
\usepackage[spanish,activeacute]{babel}
\usepackage[utf8]{inputenc}
\usepackage[T1]{fontenc} %Necesario para el uso de las comillas latinas.

%Código
\usepackage{listings}
\usepackage{color}

\definecolor{dkgreen}{rgb}{0,0.6,0}
\definecolor{gray}{rgb}{0.5,0.5,0.5}
\definecolor{mauve}{rgb}{0.58,0,0.82}

\lstset{frame=tb,
  aboveskip=3mm,
  belowskip=3mm,
  showstringspaces=false,
  columns=flexible,
  basicstyle={\small\ttfamily},
  numbers=left,
  numberstyle=\tiny\color{gray},
  keywordstyle=\color{blue},
  commentstyle=\color{dkgreen},
  stringstyle=\color{mauve},
  breaklines=true,
  breakatwhitespace=true,
  tabsize=3
}

\usepackage{hyperref}
\hypersetup{
  pdfauthor={Antonio Álvarez Caballero},
  pdftitle={Sistema experto de apoyo en inversión en bolsa},
  unicode,
  plainpages=false,
  colorlinks,
  citecolor=black,
  filecolor=black,
  linkcolor=black,
  urlcolor=black,
}

%Paquetes matemáticos
\usepackage{amsmath,amsfonts,amsthm}
\usepackage{amssymb}
%\usepackage[all]{xy} %Para diagramas
\usepackage{graphicx} %Inclusion imagenes
\usepackage{enumerate} %Personalización de enumeraciones
\usepackage{mathtools} %Para \coloneqq
\usepackage{tikz} %Dibujos
\usetikzlibrary{positioning} %Distancias y posicionamiento en tikz

% Algoritmos
\usepackage{algorithm}
\usepackage[noend]{algpseudocode}

% \usepackage{algorithm}% http://ctan.org/pkg/algorithms
% \usepackage{algcompatible}% http://ctan.org/pkg/algorithmicx

%Tipografía escalable
\usepackage{lmodern}
%Legibilidad
\usepackage{microtype}

\title{Sistema experto de apoyo en inversión en bolsa}
\author{Antonio Álvarez Caballero $\cdot$ 15457968J\\
    \href{mailto:analca3@correo.ugr.es}{analca3@correo.ugr.es}}
\date{Curso 2015 - 2016}

\theoremstyle{definition}
\newtheorem{regla}{Regla}

% Useful for commenting align environments!
% Taken from http://tex.stackexchange.com/a/38171
\newcommand{\comment}[1]{%
  \text{\phantom{(#1)}} \tag{#1}
}

\usepackage{hyperref}

% Comando para añadir URL, enlace y pie de página de una vez.
\newcommand\fnurl[2]{%
  \href{#2}{#1}\footnote{\url{#2}}%
}

%%%%%%%%%%%%%%
\begin{document}
  \maketitle
  \tableofcontents
  \newpage

  \section{Resumen del funcionamiento del sistema}

  Se ha desarrollado un sistema experto como apoyo a la inversión en bolsa.
  Utilizando el estado actual del mercado, las noticias que afectan a éste y la
  cartera actual del usuario, se ofrecen cinco operaciones de tres tipos.
  Comprar acciones que se espera que se obtenga alta rentabilidad, vender las
  que se espera que tengan baja rentabilidad y el intercambio de acciones
  entre un valor de la cartera y uno de mercado. Todo esto para aumentar la
  rentabilidad de las acciones del usuario.

  Como cualquier sistema experto, es un sistema de apoyo en la que el usuario
  tiene siempre la decisión. Además se permite no sólo decidir la acción sino
  en qué cantidad.

  Como funcionalidad adicional se permite guardar el estado de la cartera en
  disco para que el usuario pueda guardar y retomar su trabajo.

  \section{Proceso seguido en el desarrollo}
  Teniendo en cuenta que el experto, directivo y usuarios eran la misma persona,
  el desarrollo del sistema ha seguido lo explicado en clase de teoría: estudio
  del problema, reuniones iniciales con el experto para la primera toma de
  contacto, reuniones posteriores para detallar y contactos con directivos
  y usuarios para definir las entradas y salidas del sistema. Además, se ha
  llevado a cabo un proceso de verificación y validación con técnicas vistas
  en clase de teoría.

  \subsection{Sesiones con el experto}
  Hubo un total de tres sesiones con el experto, y el conocimiento restante se
  extrajo de un documento redactado por el propio experto.

  En la primera reunión, en la que el experto actuó como directivo y usuario
  también, se definieron pinceladas de la entrada y salida del sistema, además
  de las primeras tareas del sistema (detección de valores estables y peligrosos).
  Además se empezó a obtener información acerca de cómo el usuario quería que
  actuara el sistema y cómo funcionaría en general. Aquí se estableció que
  se debían mostrar cinco propuestas y de dónde se sacaban los datos de entrada.

  En la segunda sesión se comenzaron a detallar el funcionamiento de cada módulo
  y la estructura que debían seguir. Se dieron ideas generales de cómo debía
  funcionar el módulo de detección de valores peligrosos, dando además algunas
  reglas para el mismo. Además se comenzó a estructurar el sistema.

  En la tercera y última sesión se definieron algunos conocimientos sobre el
  ámbito, se dieron nuevas reglas y se llegó a la conclusión de que el conocimiento
  del experto estaba mejor escrito. Así que el experto redactó un documento
  al que hemos tenido acceso durante el desarrollo.

  \subsection{Procedimiento de verificación y validación}
  Para la verifiación y validación del sistema se han utilizado las técnicas vistas
  en clase de teoría, además de las técnicas clásicas de ingeniería del software.

  \subsubsection*{Verificación}
  La verificación es el primer paso, que es comprobar que el código sea correcto.

  Se debe estudiar la consistencia del sistema, para la que se crean ejemplos
  ficticios y se intenta hacer que el programa se salga de su flujo usual. Un
  resultado clave de este estudio fue que se debía tener control en todo momento
  del módulo que estaba trabajando, para que no existieran solapamientos entre
  reglas o desorden de actuación. Para ello se define un hecho \emph{Modulo i},
  con $i$ el número del módulo. Cuando se cambia de módulo se borra este hecho
  y se incluye uno nuevo con el número de módulo siguiente.

  Se analizó la completitud del sistema comparando el documento del experto
  con el código, realizando pruebas para detectar posibles fallos.

  \subsubsection*{Validación}
  La validación es la comprobación de que el sistema realiza lo que el usuario
  necesita.

  Primero se realizó un análisis de las reglas implementadas para asegurarnos
  que no había ninguna mal expresada, siempre utilizando el documento del experto
  como referencia.

  Se realizaron tests para verificar que todo cumple lo que se pide: adición de
  valores ficticios a la cartera, intentar comprar más acciones de las que se
  pueden... Así se decidió aplicar prioridades a las reglas para que las noticias
  buenas influyeran más que las malas.

  Se ha llevado a cabo una validación interpretativa, donde el ingeniero de
  conocimiento ha actuado como usuario para pulir la interacción del sistema.
  Así se ha intentado escoger una interfaz sencilla y con instrucciones claras
  para que el usuario no tenga ningún problema al utilizar el sistema.

  
  \section{Descripción detallada del sistema}
  \subsection{Variables de entrada del problema}
  \subsection{Conocimiento global del sistema}
  \subsection{Especificación de los módulos}
  \subsubsection{Módulo de lectura y detección de valores inestables}
  \subsubsection*{Estructura del conocimiento}
  \subsubsection*{Lectura de los datos de entrada}
  \subsubsection*{Detección de valores estables o inestables}
  \subsubsection{Módulo de valores peligrosos}
  \subsubsection{Módulo de valores sobrevalorados}
  \subsubsection{Módulo de valores infravalorados}
  \subsubsection{Módulo de obtención de propuestas}
  \subsubsection{Módulo de interacción con el usuario}
  \section{Manual de uso del sistema}
  \subsection{Estado actual de la cartera}
  \subsection{Propuestas de movimientos}
  \subsection{Actualización de la cartera en disco}
  \subsection{Terminar el sistema}














\end{document}
