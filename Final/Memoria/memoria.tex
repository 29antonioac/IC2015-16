\documentclass[a4paper, 11pt, titlepage]{article}
%Comandos para configurar el idioma
\usepackage[spanish,activeacute]{babel}
\usepackage[utf8]{inputenc}
\usepackage[T1]{fontenc} %Necesario para el uso de las comillas latinas.

%Código
\usepackage{listings}
\usepackage{color}

\definecolor{dkgreen}{rgb}{0,0.6,0}
\definecolor{gray}{rgb}{0.5,0.5,0.5}
\definecolor{mauve}{rgb}{0.58,0,0.82}

\lstset{frame=tb,
  aboveskip=3mm,
  belowskip=3mm,
  showstringspaces=false,
  columns=flexible,
  basicstyle={\small\ttfamily},
  numbers=left,
  numberstyle=\tiny\color{gray},
  keywordstyle=\color{blue},
  commentstyle=\color{dkgreen},
  stringstyle=\color{mauve},
  breaklines=true,
  breakatwhitespace=true,
  tabsize=3
}

\usepackage{hyperref}
\hypersetup{
  pdfauthor={Antonio Álvarez Caballero},
  pdftitle={Sistema experto de apoyo en inversión en bolsa},
  unicode,
  plainpages=false,
  colorlinks,
  citecolor=black,
  filecolor=black,
  linkcolor=black,
  urlcolor=black,
}

%Paquetes matemáticos
\usepackage{amsmath,amsfonts,amsthm}
\usepackage{amssymb}
%\usepackage[all]{xy} %Para diagramas
\usepackage{graphicx} %Inclusion imagenes
\usepackage{enumerate} %Personalización de enumeraciones
\usepackage{mathtools} %Para \coloneqq
\usepackage{tikz} %Dibujos
\usetikzlibrary{positioning} %Distancias y posicionamiento en tikz

% Algoritmos
\usepackage{algorithm}
\usepackage[noend]{algpseudocode}

% \usepackage{algorithm}% http://ctan.org/pkg/algorithms
% \usepackage{algcompatible}% http://ctan.org/pkg/algorithmicx

%Tipografía escalable
\usepackage{lmodern}
%Legibilidad
\usepackage{microtype}

\title{Sistema experto de apoyo en inversión en bolsa}
\author{Antonio Álvarez Caballero $\cdot$ 15457968J\\
    \href{mailto:analca3@correo.ugr.es}{analca3@correo.ugr.es}}
\date{Curso 2015 - 2016}

\theoremstyle{definition}
\newtheorem{regla}{Regla}

% Useful for commenting align environments!
% Taken from http://tex.stackexchange.com/a/38171
\newcommand{\comment}[1]{%
  \text{\phantom{(#1)}} \tag{#1}
}

\usepackage{hyperref}

% Comando para añadir URL, enlace y pie de página de una vez.
\newcommand\fnurl[2]{%
  \href{#2}{#1}\footnote{\url{#2}}%
}

%%%%%%%%%%%%%%
\begin{document}
  \maketitle
  \tableofcontents
  \newpage

  \section{Resumen del funcionamiento del sistema}

  Se ha desarrollado un sistema experto como apoyo a la inversión en bolsa.
  Utilizando el estado actual del mercado, las noticias que afectan a éste y la
  cartera actual del usuario, se ofrecen cinco operaciones de tres tipos.
  Comprar acciones que se espera que se obtenga alta rentabilidad, vender las
  que se espera que tengan baja rentabilidad y el intercambio de acciones
  entre un valor de la cartera y uno de mercado. Todo esto para aumentar la
  rentabilidad de las acciones del usuario.

  Como cualquier sistema experto, es un sistema de apoyo en la que el usuario
  tiene siempre la decisión. Además se permite no sólo decidir la acción sino
  en qué cantidad.

  Como funcionalidad adicional se permite guardar el estado de la cartera en
  disco para que el usuario pueda guardar y retomar su trabajo.

  \section{Proceso seguido en el desarrollo}
  Teniendo en cuenta que el experto, directivo y usuarios eran la misma persona,
  el desarrollo del sistema ha seguido lo explicado en clase de teoría: estudio
  del problema, reuniones iniciales con el experto para la primera toma de
  contacto, reuniones posteriores para detallar y contactos con directivos
  y usuarios para definir las entradas y salidas del sistema. Además, se ha
  llevado a cabo un proceso de verificación y validación con técnicas vistas
  en clase de teoría.

  \subsection{Sesiones con el experto}
  Hubo un total de tres sesiones con el experto, y el conocimiento restante se
  extrajo de un documento redactado por el propio experto.

  En la primera reunión, en la que el experto actuó como directivo y usuario
  también, se definieron pinceladas de la entrada y salida del sistema, además
  de las primeras tareas del sistema (detección de valores estables y peligrosos).
  Además se empezó a obtener información acerca de cómo el usuario quería que
  actuara el sistema y cómo funcionaría en general. Aquí se estableció que
  se debían mostrar cinco propuestas y de dónde se sacaban los datos de entrada.

  En la segunda sesión se comenzaron a detallar el funcionamiento de cada módulo
  y la estructura que debían seguir. Se dieron ideas generales de cómo debía
  funcionar el módulo de detección de valores peligrosos, dando además algunas
  reglas para el mismo. Además se comenzó a estructurar el sistema.

  En la tercera y última sesión se definieron algunos conocimientos sobre el
  ámbito, se dieron nuevas reglas y se llegó a la conclusión de que el conocimiento
  del experto estaba mejor escrito. Así que el experto redactó un documento
  al que hemos tenido acceso durante el desarrollo.

  \subsection{Procedimiento de verificación y validación}
  Para la verifiación y validación del sistema se han utilizado las técnicas vistas
  en clase de teoría, además de las técnicas clásicas de ingeniería del software.

  \subsubsection*{Verificación}
  La verificación es el primer paso, que es comprobar que el código sea correcto.

  Se debe estudiar la consistencia del sistema, para la que se crean ejemplos
  ficticios y se intenta hacer que el programa se salga de su flujo usual. Un
  resultado clave de este estudio fue que se debía tener control en todo momento
  del módulo que estaba trabajando, para que no existieran solapamientos entre
  reglas o desorden de actuación. Para ello se define un hecho \emph{Modulo i},
  con $i$ el número del módulo. Cuando se cambia de módulo se borra este hecho
  y se incluye uno nuevo con el número de módulo siguiente.

  Se analizó la completitud del sistema comparando el documento del experto
  con el código, realizando pruebas para detectar posibles fallos.

  \subsubsection*{Validación}
  La validación es la comprobación de que el sistema realiza lo que el usuario
  necesita.

  Primero se realizó un análisis de las reglas implementadas para asegurarnos
  que no había ninguna mal expresada, siempre utilizando el documento del experto
  como referencia.

  Se realizaron tests para verificar que todo cumple lo que se pide: adición de
  valores ficticios a la cartera, intentar comprar más acciones de las que se
  pueden... Así se decidió aplicar prioridades a las reglas para que las noticias
  buenas influyeran más que las malas.

  Se ha llevado a cabo una validación interpretativa, donde el ingeniero de
  conocimiento ha actuado como usuario para pulir la interacción del sistema.
  Así se ha intentado escoger una interfaz sencilla y con instrucciones claras
  para que el usuario no tenga ningún problema al utilizar el sistema.


  \section{Descripción detallada del sistema}
  Se detalla el funcionamiento del sistema, además de su estructura y la
  interacción entre módulos.

  \subsection{Variables de entrada del problema}
  El sistema tiene las siguientes variables de entrada:

    \begin{enumerate}
        \item \textbf{Cartera de valores}: Archivo de texto con información sobre la cartera de valores del usuario. Cada línea del archivo se corresponde con un valor en el que el usuario tenga inversión y está compuesta por los siguientes campos, delimitados por espacios:
        \begin{itemize}
            \item Nombre del valor: cadena de caracteres que identifica de forma unívoca un valor.
            \item Número de acciones: cantidad de acciones de esa empresa que tiene el usuario.
            \item Valor: cantidad de dinero que valen las acciones; este número ---en euros--- debe coincidir con el producto \code{número de acciones} $\times$ \code{precio de cada acción}.
        \end{itemize}
        Este archivo debe contener al menos una línea con el nombre \code{DISPONIBLE}, que informa de la cantidad de dinero líquido disponible, indicada en el campo \code{Valor}. El campo \code{número de acciones} de esta línea es irrelevante, aunque suele coincidir con el de \code{Valor}.\footnote{De hecho, en todo el sistema se ignora el campo \code{Número de acciones} de la línea cuyo nombre es \code{DISPONIBLE}, trabajando sobre el campo \code{Valor} y modificando única y exclusivamente este.}

        \item \textbf{Análisis de los valores del mercado}: Archivo de texto con información sobre el estado actual de  los valores del Ibex35. Cada línea del archivo se corresponde con un valor del Ibex35 y está compuesta por los siguientes campos, delimitados por espacios:
        \begin{itemize}
            \item Nombre: cadena de caracteres que identifica de forma unívoca un valor.
            \item Precio: precio ---en euros--- de una acción.
            \item VarDia: variación del valor con respecto al día anterior ---en tanto por ciento---.
            \item Capitalización: valor total de la empresa ---en miles de euros---.
            \item Ratio precio-beneficio (PER): capitalización dividida por los beneficios anuales obtenidos por la empresa ---en tanto por ciento---.
            \item Rentabilidad por dividendo (RPD): ratio que indica la cantidad recibida por los accionistas por dividendos ---en tanto por uno---.
            \item Tam: cadena de caracteres que indica el tamaño de la empresa: \code{GRANDE}, \code{MEDIANA} o \code{PEQUENIA}.
            \item Ibex: capitalización con respecto al total del Ibex ---en tanto por ciento---.
            \item EtiqPER: cadena de caracteres que indica el estado actual del PER: \code{Alto}, \code{Medio} o \code{Bajo}.
            \item EtiqRPD: cadena de caracteres que indica el estado actual del RPD: \code{Alto}, \code{Medio} o \code{Bajo}.
            \item Sector: cadena de caracterres que identifica de forma unívoca el sector al que pertenece la empresa.
            \item Var5Dias: variación del precio respecto al que tenía hace cinco días ---en tanto por ciento---.
            \item Perd3Consec: valor booleano que indica si la empresa ha tenido pérdidas durante los últimos tres días ---\code{true}--- o no ---\code{false}---.
            \item Perd5Consec: valor booleano que indica si la empresa ha tenido pérdidas durante los últimos cinco días ---\code{true}--- o no ---\code{false}---.
            \item VarRelativaSector: variación de los últimos cinco días relativa a la variación del sector en el mismo tramo temporal ---en tanto por ciento---.
            \item VarRelativaSectorChico: variable booleana que indica si la variación anterior es menor de un -5\% ---\code{true}--- o no ---\code{false}---.
            \item VarMensual: variación del valor con respecto al mes anterior ---en tanto por ciento---.
            \item VarTrimestral: variación del valor con respecto al trimestre anterior ---en tanto por ciento---.
            \item VarSemestral: variación del valor con respecto al semestre anterior ---en tanto por ciento---.
            \item VarAnual: variación del valor con respecto al año anterior ---en tanto por ciento---.
        \end{itemize}

        \item \textbf{Análisis de los sectores del mercado}: Archivo de texto con información sobre el estado actual de los sectores del Ibex35. Cada línea del archivo se corresponde con un sector del Ibex35 y está compuesta por los siguientes campos, delimitados por espacios:
        \begin{itemize}
            \item Nombre: cadena de caracteres que identifica de forma unívoca un sector.
            \item VarDia: variación del valor con respecto al día anterior ---en tanto por ciento---.
            \item Capitalización: valor total de la empresa ---en miles de euros---.
            \item Ratio precio-beneficio (PER): capitalización dividida por los beneficios anuales obtenidos por la empresa ---en tanto por ciento---.
            \item Rentabilidad por dividendo (RPD): ratio que indica la cantidad recibida por los accionistas por dividendos ---en tanto por uno---.
            \item Ibex: capitalización con respecto al total del Ibex ---en tanto por ciento---.
            \item Var5Dias: variación del precio respecto al que tenía hace cinco días ---en tanto por ciento---.
            \item Perd3Consec: valor booleano que indica si la empresa ha tenido pérdidas durante los últimos tres días ---\code{true}--- o no ---\code{false}---.
            \item Perd5Consec: valor booleano que indica si la empresa ha tenido pérdidas durante los últimos cinco días ---\code{true}--- o no ---\code{false}---.
            \item VarMensual: variación del valor con respecto al mes anterior ---en tanto por ciento---.
            \item VarTrimestral: variación del valor con respecto al trimestre anterior ---en tanto por ciento---.
            \item VarSemestral: variación del valor con respecto al semestre anterior ---en tanto por ciento---.
            \item VarAnual: variación del valor con respecto al año anterior ---en tanto por ciento---.
        \end{itemize}
        \item \textbf{Análisis de la situación político-económica}: archivo de texto con información sobre las noticias relevantes al mercado. Cada línea del archivo se corresponde con una noticia y está compuesta por los siguientes campos, delimitados por espacios:
        \begin{itemize}
            \item Nombre: cadena de caracteres que indica el valor del Ibex35 al que hace referencia la noticia ---si la noticia es de carácter general sobre el estado de la economía, este campo contiene la cadena \code{ECONOMIA}---.
            \item Valoración: cadena de caracteres que indica la valoración que hace la noticia sobre la empresa a la que hace referencia; puede ser \code{Buena} o \code{Mala}.
            \item Antigüedad: número de días transcurridos desde la publicación de la noticia.
        \end{itemize}
    \end{enumerate}

  \subsection{Variables de salida del problema}

  El sistema produce un único tipo de salida: una serie de cinco transacciones propuestas, ordenadas de mayor a menor rentabilidad esperada. Cada propuesta se compone de los siguientes apartados:
  \begin{itemize}
      \item Acción de la propuesta: aquí se indica qué se propone ---comprar, vender o intercambiar--- y cuáles son las empresas involucradas en la transacción.
      \item Rentabilidad esperada: porcentaje de rentabilidad esperada tras ejecutar la transacción propuesta.
      \item Razón: se da un razonamiento exhaustivo de por qué se hace la propuesta, en qué se basa la catalogación de los valores y qué hechos han llevado al sistema a proponer la transacción.
  \end{itemize}

  Además, el sistema es capaz de simular la ejecución de las propuestas elegidas por el usuario y guardar en un fichero de texto, si el usuario así lo desea y con el formato indicado en el primer punto de la sección \ref{ch:varEntrada}, el estado de la cartera tras aplicar las propuestas.

  \subsection{Conocimiento global del sistema}
  El sistema obtiene su conocimiento en el \emph{Módulo 0}. Aquí se definen
  estructuras para la base del conocimiento, además de generarla utilizando
  sendos ficheros de texto.

  \subsection{Especificación de los módulos}
  Aquí detallaremos la estructura general del sistema en módulos:

    \begin{enumerate}\addtocounter{enumi}{-1}
        \item Módulo de lectura y detección de valores inestables.
        \item Módulo de detección de valores peligrosos.
        \item Módulo de detección de valores sobrevalorados.
        \item Módulo de detección de valores infravalorados.
        \item Módulo de obtención de propuestas.
        \item Módulo de interacción con el usuario.
    \end{enumerate}

    Como se ha mencionado antes, la gestión de módulos se ha realizado utilizando
    un hecho \emph{Módulo i} con $i$ el número de módulo. Al salir de un módulo
    se borra este hecho y se añade uno con un número superior. Hay que añadir
    el 0 para que todo se comience a ejecutar.

  \subsubsection{Módulo de lectura y detección de valores inestables}
  Este fichero tiene como objetivo leer los datos de los ficheros de texto
  y detectar los valores inestables a partir de las variables de entrada.

  \subsubsection*{Estructura del conocimiento}
  Además de los campos ofrecidos por el experto, se añaden el \emph{RPA}, que es
  el rendimiento por año esperado (calculado utilizando la variación trimestral);
  la estabilidad (indicando si el valor es estable); y si el valor es sobrevalorado,
  infravalorado o no tiene valoración. En el caso de tenerla tiene su explicación
  asociada.

  Para los valores de la cartera se añade un campo más, peligroso, que vale \emph{true}
  si dicho valor es peligroso.

  Se han definido también plantillas para modelar propuestas, en las que se
  almacenan las operaciones a realizar y sobre qué empresa/s.

  \subsubsection*{Lectura de los datos de entrada}
  El módulo 0 se encarga de leer los ficheros de entrada. Se definen hechos
  con las rutas relativas de los ficheros a leer, además de sendas reglas para
  leer cada tipo de fichero.

  Además, se calcula el RPA tal y como se ha descrito anteriormente.

  \subsubsection*{Detección de valores estables o inestables}
  En el módulo 0 también se detectan los valores inestables, utlizando estas reglas.

  \begin{regla}
        Si un valor $V$ pertenece al sector de la construcción, entonces $V$ es inestable.
    \end{regla}

    \begin{regla}
        Si la economía está bajando ---el Ibex tiene pérdidas en los cinco últimos días--- y un valor $V$ pertenece al sector servicios, entonces $V$ es inestable.
    \end{regla}

    \begin{regla}
        Si hay una noticia positiva sobre un valor $V$ o su sector y la noticia tiene menos de dos días de antigüedad, entonces $V$ es estable.
    \end{regla}

    \begin{regla}
        Si hay una noticia negativa sobre un valor $V$, sobre su sector o sobre la economía en general y la noticia tiene menos de dos días de antigüedad, entonces $V$ es inestable.
    \end{regla}


  \subsubsection{Módulo de valores peligrosos}
  El módulo 1 define las reglas para detectar valores peligrosos.

  \begin{regla}
        Si un valor $V$ de la cartera es inestable y ha tenido pérdidas durante los tres últimos días, entonces $V$ es peligroso.
    \end{regla}

    \begin{regla}
        Si un valor $V$ de la cartera ha tenido pérdidas durante los últimos cinco días y su caída con respecto a la del sector es mayor a un 5\%, entonces $V$ es peligroso.\footnote{El experto describió esta regla en el documento de una forma más ambigua, pero tras contactos posteriores con él se llegó al acuerdo de que la forma aquí expuesta ---es decir, que el campo \code{VarRelativaSectorChico} sea igual a \code{false}--- era la correcta.}
    \end{regla}
  \subsubsection{Módulo de valores sobrevalorados}

  El módulo 2 calcula los valores sobrevalorados.
  \begin{regla}
        Dado un valor $V$, si su PER está marcado como \code{Alto} y su RPD como \code{Bajo}, entonces $V$ está sobrevalorado.
    \end{regla}

    \begin{regla}
        Dado un valor $V$ cuyo tamaño sea \code{PEQUENIA}:
        \begin{itemize}
            \item Si su PER está marcado como \code{Alto}, entonces $V$ está sobrevalorado.
            \item Si su PER está marcado como \code{Medio} y su RPD como \code{Bajo}, entonces $V$ está sobrevalorado.
        \end{itemize}
    \end{regla}

    \begin{regla}
        Dado un valor $V$ cuyo tamaño sea \code{GRANDE}:
        \begin{itemize}
            \item Si su RPD está marcado como \code{Bajo} y su PER como \code{Medio} o \code{Alto}, entonces $V$ está sobrevalorado.
            \item Si su RPD está marcado como \code{Medio} y su PER como \code{Alto}, entonces $V$ está sobrevalorado.
        \end{itemize}
    \end{regla}


  \subsubsection{Módulo de valores infravalorados}

  El módulo 3 calcula los valores infravalorados del sistema.

  \begin{regla}
      Si un valor $V$ tiene el PER marcado como \code{Bajo} y el RPD como \code{Alto}, entonces $V$ está infravalorado.
  \end{regla}

  \begin{regla}
      Si un valor $V$ tiene el PER marcado como \code{Bajo}, ha tenido una caída de más de un 30\% en los últimos 3, 6 o 12 meses y ha subido entre un 0\% y un 10\% en el último mes, entonces $V$ está infravalorado.
  \end{regla}

  \begin{regla}
      Si un valor $V$ tiene un tamaño \code{GRANDE}, su RPD está marcado como \code{Alto} y su PER como \code{Medio}, su variación mensual es mayor o igual al 0\% y tiene una variación respecto del sector positiva, entonces $V$ está infravalorado.
  \end{regla}

  El razonamiento termina aquí, pasando a los módulos 4 y 5 donde se recopilan
  y presentan propuestas al usuario.

  \subsubsection{Módulo de obtención de propuestas}

  En esete módulo se obtienen las propuestas para ofrecer al usuario.

  \begin{regla}
        Si un valor $V$ de la cartera actual es peligroso, en el último mes ha tenido pérdidas, y en ese mismo tramo temporal su caída con respecto a la de su sector es de más de un 3\%, entonces se debe proponer vender las acciones de $V$.

        La rentabilidad esperada de esta propuesta es:
        \[
        RE = 20 - RPD
        \]
    \end{regla}

    \begin{regla}
        Si un valor $V$ está infravalorado y el usuario tiene dinero para comprar al menos una acción de $V$, entonces se debe proponer comprar acciones de $V$.

        La rentabilidad esperada de esta propuesta es:
        \[
        RE = 100 \frac{PER_{sector} - PER}{5 PER} + RPD
        \]
        donde $PER_{sector}$ es el $PER$ del sector al que pertenece $V$.
    \end{regla}

    \begin{regla}
        Si un valor $V$ de la cartera actual está sobrevalorado y su RPA es menor que $5 + PrecioDinero$, entonces se debe proponer vender las acciones de $V$.

        La rentabilidad esperada de esta propuesta es:
        \[
        RE = \frac{PER - PER_{sector}}{5 PER} - RPD
        \]
        donde $PER_{sector}$ es el $PER$ del sector al que pertenece $V$.
    \end{regla}

    \begin{regla}
        Si un valor $V_1$ no está sobrevalorado, un valor $V_2$ de la cartera actual no está infravalorado y el $RPD_1$ de $V_1$ es mayor que la suma $RPA_2 + RPD_2 + 1$ ---donde $RPA_2$ y $RPD_2$ son los campos relativos al valor $V_2$---, entonces se debe proponer cambiar las acciones de $V_2$ por acciones de $V_1$.

        La rentabilidad esperada de esta propuesta es:
        \[
        RE = RPD_1 - (RPA_2 + RPD_2 + 1)
        \]
    \end{regla}

  \subsubsection{Módulo de interacción con el usuario}

  El módulo 5 interactúa con el usuario mostrando las 5 mejores propuestas.
  Lo hace utilizando un menú y con la base del conocimiento inducida. 


  \section{Manual de uso del sistema}
  \subsection{Estado actual de la cartera}
  \subsection{Propuestas de movimientos}
  \subsection{Actualización de la cartera en disco}
  \subsection{Terminar el sistema}














\end{document}
